\section{动态交互}

D3 支持制作动态的图表。有时候,图表的变化需要缓慢地发生,以便于让用户看清楚变化的过程。此外,用户还可能会对图表中的部分元素进行点击,而图表可会对不同的事件作出反应。D3提供了这些能提升交互式的用户体验的方法和操作。

\subsection{什么是动态效果}

前面几节中制作的图表是直接显示出现,且绘制完成后不再发生变化的,这是静态的图表。

动态的图表,是指图表在某一时间段会发生某种变化,可能是形状、颜色、位置等,用户能够看到变化的过程。

例如,有一个圆,圆心坐标为 (100, 100)。现在我们希望圆的 x 坐标从 100 移到 300,并且移动过程在 2 秒的时间内发生。

这种时候就需要用到动态效果,在 D3 里我们称之为过渡(transition)。

\subsection{实现动态的方法}

D3 提供了 4 个方法用于实现图形的过渡:从状态 A 变为状态 B。

\subsubsection{transition()}

\verb|transition()| 用于启动过渡效果。其前后是图形变化前后的状态(形状、位置、颜色等)。例如这段代码片段:

\begin{minted}[frame=lines,framesep=2mm,baselinestretch=1.2,fontsize=\footnotesize,linenos]{javascript}
.attr("fill", "red") // 初始颜色为红色
.transition() // 启动过渡
.attr("fill", "green") // 终止颜色为绿色
\end{minted}

D3 会自动对两种颜色(如上例中的红色和绿色)之间的颜色值(RGB值)进行插值计算,得到过渡用的颜色值。我们可以观察到颜色变化时的动态渐变效果。

\subsubsection{duration()}

\verb|duration()| 用来指定过渡的持续时间,单位为毫秒。

\begin{minted}[frame=lines,framesep=2mm,baselinestretch=1.2,fontsize=\footnotesize,linenos]{javascript}
.attr("fill", "red") // 初始颜色为红色
.transition() // 启动过渡
.duration(1000) // 设置过渡的持续时间为1秒
.attr("fill", "green") // 终止颜色为绿色
\end{minted}

\subsubsection{delay()}

\verb|delay()| 指定延迟的时间,表示一定时间后才开始转变,单位同样为毫秒。此函数可以对整体指定延迟,也可以对个别指定延迟。

如对整体指定延时:

\begin{minted}[frame=lines,framesep=2mm,baselinestretch=1.2,fontsize=\footnotesize,linenos]{javascript}
.transition()
.duration(1000)
.delay(500)
\end{minted}

图形整体会在延迟 500 毫秒后发生变化,变化的时长为 1000 毫秒。因此,过渡的总时长为1500毫秒。

也能对一个个的图形(假设已经绑定了数据)分别指定设置延时:

\begin{minted}[frame=lines,framesep=2mm,baselinestretch=1.2,fontsize=\footnotesize,linenos]{javascript}
.transition()
.duration(1000)
.delay(funtion(d,i){
    return 200*i;
})
\end{minted}

假设有10个元素,则第1个元素延时0毫秒,第2个元素延时200毫秒,第3个元素延时400毫秒,以此类推。

\subsubsection{ease()}

\verb|ease()| 指定过渡的缓动函数。常用的有:

\begin{enumerate}
    \item \verb|d3.easeLinear|:普通的线性变化
    \item \verb|d3.easeCircle|:慢慢地到达变换的最终状态
    \item \verb|d3.easeElastic|:带有弹跳的到达最终状态
    \item \verb|d3.easeBounce|:在最终状态处弹跳几次
\end{enumerate}

调用方式形如 \verb|.ease(d3.easeLinear)|。

\subsection{什么是交互}

交互,指的是用户输入了某种指令,程序接受到指令之后必须做出某种响应。对可视化图表来说,交互能使图表更加生动,能表现更多内容。例如,拖动图表中某些图形、鼠标滑到图形上出现提示框、用触屏放大或缩小图形等等。

用户用于交互的工具一般有三种:鼠标、键盘、触屏。

\subsection{如何添加交互}

对某一元素添加交互操作十分简单,代码如下:

\begin{minted}[frame=lines,framesep=2mm,baselinestretch=1.2,fontsize=\footnotesize,linenos]{javascript}
var circle = svg.append("circle");

circle.on("click", function(){
    // 在此处添加交互内容
});
\end{minted}

这段代码在 SVG 中添加了一个圆,然后通过 \verb|on()| 添加了一个监听器。在 D3 中,每一个选择集都有 \verb|on()| 函数,用于添加事件监听器。

其中,\verb|on()| 的第一个参数是监听的事件,第二个参数是监听到事件后响应的内容,第二个参数是一个函数。

对于鼠标,常用的事件有:

\begin{enumerate}
    \item click:鼠标单击某元素时,相当于 mousedown 和 mouseup 组合在一起。
    \item mouseover:光标放在某元素上。
    \item mouseout:光标从某元素上移出来时。
    \item mousemove:鼠标被移动的时候。
    \item mousedown:鼠标按钮被按下。
    \item mouseup:鼠标按钮被松开。
    \item dblclick:鼠标双击。
\end{enumerate}

键盘常用的事件有三个:

\begin{enumerate}
    \item keydown:当用户按下任意键时触发,按住不放会重复触发此事件。该事件不会区分字母的大小写,例如“A”和“a”被视为一致。
    \item keypress:当用户按下字符键(大小写字母、数字、加号、等号、回车等)时触发,按住不放会重复触发此事件。该事件区分字母的大小写。
    \item keyup:当用户释放键时触发,不区分字母的大小写。
\end{enumerate}

触屏常用的事件有三个:

\begin{enumerate}
    \item touchstart:当触摸点被放在触摸屏上时。
    \item touchmove:当触摸点在触摸屏上移动时。
    \item touchend:当触摸点从触摸屏上拿开时。
\end{enumerate}
