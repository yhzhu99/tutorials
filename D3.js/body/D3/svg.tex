\section{绘制SVG图形}

前面我们所处理对象都是 HTML 的文字,没有涉及图形的制作。若要进行绘图,首要需要的是一块绘图的``画布''。在HTML 5中, 提供了两种强有力的``画布'':SVG 和 Canvas。其中,D3对SVG的支持非常好,提供了众多的SVG图形的生成器。本节中,我们将用SVG绘制简单的条形图为例来了解D3是如何操作SVG图形的。

\subsection{什么是SVG}

SVG,指可缩放矢量图形(Scalable Vector Graphics),是用于描述二维矢量图形的一种图形格式,是由万维网联盟制定的开放标准。 SVG 使用 XML 格式来定义图形,除了 IE8 之前的版本外,绝大部分浏览器都支持 SVG,可将 SVG 文本直接嵌入 HTML 中显示。

SVG 有如下特点:

\begin{enumerate}
    \item SVG 绘制的是矢量图,因此对图像进行放大不会失真。
    \item 基于 XML,可以为每个元素添加 JavaScript 事件处理器。
    \item 每个图形均视为对象,更改对象的属性,图形也会改变。
\end{enumerate}

\subsection{添加画布}

使用 D3 在 body 元素中添加 SVG 画布的代码如下:

\begin{minted}[frame=lines,framesep=2mm,baselinestretch=1.2,fontsize=\footnotesize,linenos]{javascript}
var width = 300;  // 画布的宽度
var height = 300;   // 画布的高度
var svg = d3.select("body") // 选择文档中的body元素
    .append("svg") // 添加一个SVG元素
    .attr("width", width) // 设定宽度
    .attr("height", height); // 设定高度
\end{minted}

有了画布,接下来就可以在画布上作图了。

\subsection{绘制矩形}

在 SVG 中,矩形的元素标签是 rect,在HTML中可作为标签使用。

\begin{minted}[frame=lines,framesep=2mm,baselinestretch=1.2,fontsize=\footnotesize,linenos]{html}
<svg>
    <rect></rect>
    <rect></rect>
</svg>
\end{minted}

上面的 rect 里没有矩形的属性。矩形的属性,常用的有四个:

\begin{enumerate}
    \item x:矩形左上角的 x 坐标
    \item y:矩形左上角的 y 坐标
    \item width:矩形的宽度
    \item height:矩形的高度
\end{enumerate}

要注意的是,在 SVG 中,x 轴的正方向是水平向右,y 轴的正方向是垂直向下的。

现在,我们给出一组数据(截止到2021年7月31日的东京奥运会金牌榜前五名的金牌数)对此进行可视化。数据如下:

\begin{minted}[frame=lines,framesep=2mm,baselinestretch=1.2,fontsize=\footnotesize,linenos]{javascript}
var arr = [21, 17, 16, 11, 10];  // 数据(表示矩形的宽度)
\end{minted}

我们直接将数值的大小作为矩形的宽度,然后添加以下代码:

\begin{minted}[frame=lines,framesep=2mm,baselinestretch=1.2,fontsize=\footnotesize,linenos]{javascript}
var rectHeight = 25; // 每个矩形所占的像素高度(包括空白)
svg.selectAll("rect")
    .data(arr)
    .enter()
    .append("rect")
    .attr("x",20)
    .attr("y",function(d,i){
         return i * rectHeight;
    }) // 为各元素的属性赋值
    .attr("width",function(d){
         return d*10;
         // 为了显示效果对数值进行了缩放
         // 更好的方法是通过设置比例尺来优化
    })
    .attr("height",rectHeight-5)
    .attr("fill","gold"); // 设置填充色为金色
\end{minted}

其中便应用到了\verb|enter()|方法,它使得在有数据,而没有足够图形元素的情况下,补充足够的元素。

最终结果如\figref{fig:gold_bar_chart}所示。

\begin{figure}[htbp]
    \centering
    \includegraphics[width=0.6\textwidth]{figure/D3/gold_bar_chart.png}
    \caption{\textbf{金牌榜条形图}}
    \label{fig:gold_bar_chart}
\end{figure}

\subsection{使用比例尺}

在先前的代码注释中,提到了比例尺的概念。比例尺中有线性比例尺(连续)、序数比例尺(离散)等多种类型之分。其中线性比例尺能使数值从一个连续的区间(定义域 domain)映射到另一个区间(值域 range),来解决条形图宽度的问题。

仍然对于金牌榜的数据,我们应用以下代码来使用线性比例尺:

\begin{minted}[frame=lines,framesep=2mm,baselinestretch=1.2,fontsize=\footnotesize,linenos]{javascript}
var arr = [21, 17, 16, 11, 10];
var max = d3.max(arr);
var linear = d3.scaleLinear()
        .domain([0, max]) // 定义域
        .range([0, 300]); // 值域
\end{minted}

其中\verb|d3.scaleLinear()| 的返回值,可被当做函数来使用。因此,有如这样的用法:\verb|linear(0.9)|去调用该比例尺,于是先前代码中的\verb|return d*10;|可替换成\verb|return linear(d);|
